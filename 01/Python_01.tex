% LaTeX Präsentationsvorlage (2013) der TU Graz, rev12, 2013/01/31
% !TeX encoding = UTF-8
\documentclass{beamer}
% \documentclass[aspectratio=169]{beamer}
% \usetheme{tugraz2013}
% \usetheme[notes]{tugraz2013}
\usepackage{../common/beamerthemetugraz2013}
\usepackage{color}
\usepackage{multicol}
\usepackage{bbding}
\usepackage{wasysym}
\usepackage{caption}
% \usepackage{minted}

\usepackage{listings}
\usepackage{xcolor}

\definecolor{codegreen}{rgb}{0,0.6,0}
\definecolor{codegray}{rgb}{0.5,0.5,0.5}
\definecolor{codepurple}{rgb}{0.58,0,0.82}
\definecolor{backcolour}{rgb}{0.95,0.95,0.92}
\lstdefinestyle{mystyle}{
    backgroundcolor=\color{backcolour},   
    commentstyle=\color{codegreen},
    keywordstyle=\color{magenta},
    numberstyle=\tiny\color{codegray},
    stringstyle=\color{codepurple},
    basicstyle=\ttfamily\footnotesize,
    breakatwhitespace=false,         
    breaklines=true,                 
    captionpos=b,                    
    keepspaces=true,                 
    numbers=left,                    
    numbersep=5pt,                  
    showspaces=false,                
    showstringspaces=false,
    showtabs=false,                  
    tabsize=2
}

\lstset{style=mystyle}

\usepackage{picture}
\usepackage{rotating}

\definecolor{darkred}{rgb}{0.85,0.16,0.0}
\definecolor{darkgreen}{rgb}{0.16,0.70,0.27}

\usepackage{xcolor}


\newcommand{\hrefu}[2]{\underline{\href{#1}{#2}}}
\newcommand{\hyperlinku}[2]{\underline{\hyperlink{#1}{#2}}}
\newcommand{\smallurl}[1]{%
  \begin{flushleft}
    \tiny\url{#1}
  \end{flushleft}
}
\newcommand{\smalltext}[1]{%
  \begin{flushleft}
    \tiny{#1}
  \end{flushleft}
}
\newcommand{\red}[1]{{\color{red} #1}}
\newcommand{\blue}[1]{{\color{blue} #1}}
\newcommand{\darkgreen}[1]{\textcolor{darkgreen}{#1}}
\newcommand{\darkred}[1]{\textcolor{darkred}{#1}}

\newcommand*{\vpointer}{\vcenter{\hbox{\scalebox{1.5}{\large\pointer}}}}

\newcommand{\be}[1]{\begin{equation} \label{#1}}
\newcommand{\ee}{\end{equation}}
\newcommand{\bea}[1]{\begin{eqnarray} \label{#1}}
\newcommand{\eea}{\end{eqnarray}}
\newcommand{\bean}{\begin{eqnarray*}}
\newcommand{\eean}{\end{eqnarray*}}

\newcommand{\non}{\nonumber\\}
\newcommand{\eq}[1]{(\ref{#1})}
\newcommand{\difp}[2]{\frac{\partial #1}{\partial #2}}
\newcommand{\br}{{\bf r}}
\newcommand{\bR}{{\bf R}}
\newcommand{\bA}{{\bf A}}
\newcommand{\bB}{{\bf B}}
\newcommand{\bE}{{\bf E}}
\newcommand{\bm}{{\bf m}}
%\renewcommand{\bm}{{\bf m}}
\newcommand{\bn}{{\bf n}}
\newcommand{\bN}{{\bf N}}
\newcommand{\bp}{{\bf p}}
\newcommand{\bP}{{\bf P}}
\newcommand{\bF}{{\bf F}}
\newcommand{\by}{{\bf y}}
\newcommand{\bz}{{\bf z}}
\newcommand{\bZ}{{\bf Z}}
\newcommand{\bV}{{\bf V}}
\newcommand{\bv}{{\bf v}}
\newcommand{\bu}{{\bf u}}
\newcommand{\bx}{{\bf x}}
\newcommand{\bX}{{\bf X}}
\newcommand{\bW}{{\bf W}}
\newcommand{\bJ}{{\bf J}}
\newcommand{\bj}{{\bf j}}
\newcommand{\bk}{{\bf k}}
\newcommand{\bTheta}{{\bf \Theta}}
\newcommand{\btheta}{{\boldsymbol\theta}}
\newcommand{\bOmega}{{\bf \Omega}}
\newcommand{\bomega}{{\boldsymbol\omega}}
\newcommand{\brho}{{\boldsymbol\rho}}
\newcommand{\rd}{{\rm d}}
\newcommand{\rJ}{{\rm J}}
\newcommand{\ph}{{\varphi}}
\newcommand{\te}{\theta}
\newcommand{\tht}{\vartheta}
\newcommand{\vpar}{v_\parallel}
\newcommand{\vparkb}{v_{\parallel k b}}
\newcommand{\vparkm}{v_{\parallel k m}}
\newcommand{\Jpar}{J_\parallel}
\newcommand{\ppar}{p_\parallel}
\newcommand{\Bpstar}{B_\parallel^*}
\newcommand{\intpi}{\int\limits_{0}^{2\pi}}
\newcommand{\summ}{\sum \limits_{m=-\infty}^\infty}
\newcommand{\tb}{\tau_b(\uv)}
\newcommand{\bh}{{\bf h}}
\newcommand{\cE}{{\cal E}}
\newcommand{\bsigma}{{\boldsymbol\sigma}}
\newcommand{\bS}{{\mathbf S}}
\newcommand{\bI}{{\mathbf I}}
\newcommand{\odtwo}[2]{\frac{\rd #1}{\rd #2}}
\newcommand{\pdone}[1]{\frac{\partial}{\partial #1}}
\newcommand{\pdtwo}[2]{\frac{\partial #1}{\partial #2}}
\newcommand{\ds}{\displaystyle} % commands


%% Titelblatt-Einstellungen
\title[]
{Python 01}
\author[E.~Wachmann]{\scriptsize Elias Wachmann
}
\date{2024} % \today für heutiges Datum verwenden
\institute[Institute of Theoretical and Computational Physics]
{
}
\instituteurl{www.tugraz.at}
% \institutelogo{kurz.pdf}
%~ \additionallogo{merged_logos}
\AtBeginSection[]{
  \begin{frame}
  \vfill
  \centering
  \begin{beamercolorbox}[sep=8pt,center,shadow=true,rounded=true]{title}
    \usebeamerfont{title}\insertsectionhead\par%
  \end{beamercolorbox}
  \vfill
  \end{frame}
}
%%%%%%%%%%%%%%%%%%%%%%%%%%%%%%%%%%%%%%%%%%%%%%%%%%%%%%%%%%%%%%%%%%%%%%%%%%%%
\begin{document}
%%%%%%%%%%%%%%%%%%%%%%%%%%%%%%%%%%%%%%%%%%%%%%%%%%%%%%%%%%%%%%%%%%%%%%%%%%%%
\titleframe


\begin{frame}
  \vspace*{\fill}
  \begin{center}
    \begin{quote}
        ``The magic of computing begins with 0, a simple binary digit that serves as a powerful reminder that even the smallest building blocks can create wonders.''
    \end{quote}
\end{center}
\vspace*{\fill}
\end{frame}

\begin{frame}
\frametitle{Content}
  \tableofcontents
\end{frame}

%%%%%%%%%%%%%%%%%%%%%%%%%%%%%%%%%%%%%%%%%%%%%%%%%%%%%%%%%%%%%%%%%%%%%%%%%%%%
\section{Introduction}
\begin{frame}
  \frametitle{Starting to code}
  This course is designed to give you a basic understanding of the Python programming language.\\
  But to get you all up to speed we will first have a look at the Tools you need to get started...\\
\end{frame}

\begin{frame}
  \frametitle{Tools for the course}
  \begin{itemize}
    \item Python - \hrefu{https://www.python.org/downloads/}{python.org/downloads}
    \item VsCode - \hrefu{https://code.visualstudio.com/download}{code.visualstudio.com/download}
    \item Git - \hrefu{https://git-scm.com/downloads}{git-scm.com/downloads}
    \item Gitlab - see Mail - personal account
  \end{itemize}
\end{frame}

\section{Python \& VsCode}
\begin{frame}
  \frametitle{Python in VsCode}
  Ok, so you have installed Python and VsCode.\\
  \vspace{5mm}  
  Let's take a look ...\\
\end{frame}

\section{Git \& GitLab}
\begin{frame}
  \frametitle{Git [\hrefu{https://www.youtube.com/watch?v=2sjqTHE0zok&t=1s}{video}]}
  Git is a version control system.\\
  \vspace{5mm}
  \textbf{Version control system}: A system that records changes to a file or set of files over time so that you can recall specific versions later.\\
  \vspace{5mm}
  \textbf{But why should I use it?}\\
  No more \texttt{report\_v1\_final\_final\_final\_final.pdf}!\\
  \vspace{5mm}
  You can download git \hrefu{https://git-scm.com/downloads}{here}.
\end{frame}

\begin{frame}
  \frametitle{Git -- initializes a new repository (init)}
  \textbf{Repository}: A directory where git has been initialized to start version controlling your files.\\
  \vspace{5mm}
  To initializes a new repository use \texttt{git init}.\\
  \begin{center}
    \includegraphics[width=\textwidth]{fig/git\_init.png}
   \end{center}
  This creates a hidden folder \texttt{.git} which contains all the information about the repository.\\
\end{frame}
\begin{frame}
  \frametitle{Git -- clone a repository (clone)}
  Navigate to your Gitlab repository and copy the url.\\
  \begin{minipage}[t]{0.48\textwidth}
    Click on the \texttt{clone} button and copy the \texttt{Clone with SSH} url.\\
    \end{minipage}
    \hfill 
    \begin{minipage}[t]{0.48\textwidth}
      \begin{center}
        \raisebox{-0.6\height}{\includegraphics[width=\textwidth]{fig/clone\_url.png}}
      \end{center}
  \end{minipage}
\end{frame}
\begin{frame}
  \frametitle{Git -- clone a repository (clone) cont.}
  To clone a repository use \texttt{git clone} followed by the \texttt{url} in the shell.\\
  \vspace{5mm}
  Or open the VsCode palette usind \texttt{Ctrl+Shift+P} and type \texttt{Git: Clone} followed by the \texttt{url}.\\
  \vspace{5mm}
  Choose a directory to clone the repository into.\\
  \vspace{5mm}
  \textbf{Note}: The repository is now available on your local machine.\\

\end{frame}
\begin{frame}
  \frametitle{Git -- add files (add)}
  To add files to the repository use \texttt{git add}.\\
  This stages the files for the next commit. 
  \begin{center}
    \includegraphics[width=\textwidth]{fig/git\_add.png}
   \end{center}
  Here ''Hello World'' is redirected into a file called \texttt{testfile.txt} using the \texttt{>} operator.\\
  All files in the current folder and in subfolder bellow can be added using \texttt{git add .} (with a fullstop).\\
\end{frame}
\begin{frame}
  \frametitle{Git -- see current status (status)}
  To see the current status of the repository use \texttt{git status}.\\
  \begin{center}
    \includegraphics[width=\textwidth]{fig/git\_status.png}
   \end{center}
  \texttt{testfile.txt} is staged for commit.\\
\end{frame}
\begin{frame}
  \frametitle{Git -- commit changes (commit)}
  To commit the changes use \texttt{git commit}.\\
  \begin{center}
    \includegraphics[width=\textwidth]{fig/git\_commit\_fail.png}
  \end{center}
  That didn't work $\rightarrow$ setup your git user name \& email.\\
\end{frame}
\begin{frame}
  \frametitle{Git -- setup user name \& email (config)}
  To setup your user name \& email use \texttt{git config}.\\
  The \texttt{--global} flag sets the configuration (globally) for the current user.\\
  \begin{center}
    \includegraphics[width=\textwidth]{fig/git\_commit.png}
  \end{center}
  After setting up the user name \& email you can commit the changes.\\
  Commit messages should be short and meaningful and can be added using the \texttt{-m} flag.\\
\end{frame}
\begin{frame}
  \frametitle{Git -- see commit history (log)}
  To see the commit history use \texttt{git log}.\\
  \begin{center}
    \includegraphics[width=\textwidth]{fig/git\_log.png}
  \end{center}
  Each commit has a unique identifier called a \textbf{SHA}.\\
  This allows you to go back to a specific commit.\\
\end{frame}
\begin{frame}
  \frametitle{Git -- Remote repositories}
  \textbf{Remote repository}: A repository that is hosted on the Internet or another network.\\
  \vspace{5mm}
  It is a good idea to have a remote repository to backup your work or collaborate with others.\\
  \vspace{5mm}
  \textbf{GitLab}: A website that hosts git repositories (relevant for the exercises later).\\
\end{frame}

\begin{frame}
  \frametitle{Git -- Remote (push \& pull)}
  When cloning a repository the remote alias \texttt{origin} is automatically added.\\
  All changes are loaded into the local repository.\\
  \vspace{5mm}
  After working on the project locally you can push the changes to the remote repository using \texttt{git push}.\\
  \vspace{5mm}
  \textbf{Note}: You can also pull changes from a remote repository using \texttt{git pull}.\\
\end{frame}

\begin{frame}
  \frametitle{Git -- SSH keys?}
  \textbf{SSH}: Secure Shell (SSH) is a cryptographic network protocol for operating network services securely over an unsecured network.\\
  \vspace{5mm}
  \textbf{SSH keys}: An SSH key is an access credential in the SSH protocol.\\
  \vspace{5mm}
  \textbf{You need to setup SSH keys to push changes to a remote repository.}\\
  \vspace{5mm}
  Use \hrefu{https://docs.gitlab.com/ee/user/ssh.html\#generate-an-ssh-key-pair}{this Guide} to setup \hrefu{https://docs.gitlab.com/ee/user/ssh.html\#add-an-ssh-key-to-your-gitlab-account}{link} your SSH keys to GitLab.\\
\end{frame}

\begin{frame}
  \frametitle{Create your SSH keys}
  Create the SSH keys using the \texttt{ssh-keygen} command:\\
  \lstinputlisting[language=sh]{examples/rsa.txt}
  \textbf{Windows:}
  \lstinputlisting[language=sh]{examples/rsa\_win.txt}
  \textbf{Linux:}
  \lstinputlisting[language=sh]{examples/rsa\_linux.txt}
\end{frame}
\begin{frame}
  \frametitle{Add your SSH keys to GitLab}
  Go to your GitLab account and add the SSH keys to your account.\\
  \vspace{5mm}
  Go to \texttt{Profile-Picture (upper left corner)}  $\rightarrow$ \texttt{Preferences} $\rightarrow$ \texttt{SSH Keys} $\rightarrow$ \texttt{Add SSH Key}\\
  \textcolor{red}{Copy the \textbf{\underline{whole}} content of the public key file (id\_rsa.pub) into the key field.}\\
  \vspace{5mm}
  Choose a title. Leave usage type as is. You don't have to set an expiration date.\\
  Click \texttt{Add key} - Now you are ready to push \& pull.\\
\end{frame}
\section{Additional Resources}
\begin{frame}
  \frametitle{Additional Resources}
  \begin{itemize}
    \item \hrefu{https://www.codewars.com/}{Codewars}
    \item \hrefu{https://leetcode.com/}{Leetcode}
    \item \hrefu{https://ohmygit.org/}{Oh my git!}
    \item \hrefu{https://www.learnpython.org/}{Interactive python tutorials}
    \item \hrefu{https://www.codecademy.com/learn/learn-python-3}{Learn Python 3}
    \item \hrefu{https://pll.harvard.edu/course/cs50-introduction-computer-science?delta=0}{CS50}
  \end{itemize}
   

\end{frame}

%%%%%%%%%%%%%%%%%%%%%%%%%%%%%%%%%%%%%%%%%%%%%%%%%%%%%%%%%%%%%%%%%%%%%%%%%%%%

\end{document}
%%%%%%%%%%%%%%%%%%%%%%%%%%%%%%%%%%%%%%%%%%%%%%%%%%%%%%%%%%%%%%%%%%%%%%%%%%%%

%% EOF


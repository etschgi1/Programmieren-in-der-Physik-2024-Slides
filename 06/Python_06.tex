% LaTeX Präsentationsvorlage (2013) der TU Graz, rev12, 2013/01/31
% !TeX encoding = UTF-8
\documentclass{beamer}
% \documentclass[aspectratio=169]{beamer}
% \usetheme{tugraz2013}
% \usetheme[notes]{tugraz2013}
\usepackage{../common/beamerthemetugraz2013}
\usepackage{color}
\usepackage{multicol}
\usepackage{bbding}
\usepackage{wasysym}
\usepackage{caption}
\usepackage{tikz}
\usetikzlibrary{shapes.multipart, positioning}
% \usepackage{minted}

\usepackage{listings}
\usepackage{xcolor}

\definecolor{codegreen}{rgb}{0,0.6,0}
\definecolor{codegray}{rgb}{0.5,0.5,0.5}
\definecolor{codepurple}{rgb}{0.58,0,0.82}
\definecolor{backcolour}{rgb}{0.95,0.95,0.92}
\lstdefinestyle{mystyle}{
    backgroundcolor=\color{backcolour},   
    commentstyle=\color{codegreen},
    keywordstyle=\color{magenta},
    numberstyle=\tiny\color{codegray},
    stringstyle=\color{codepurple},
    basicstyle=\ttfamily\footnotesize,
    breakatwhitespace=false,         
    breaklines=true,                 
    captionpos=b,                    
    keepspaces=true,                 
    numbers=left,                    
    numbersep=5pt,                  
    showspaces=false,                
    showstringspaces=false,
    showtabs=false,                  
    tabsize=2
}

\lstset{style=mystyle}

\usepackage{picture}
\usepackage{rotating}

\definecolor{darkred}{rgb}{0.85,0.16,0.0}
\definecolor{darkgreen}{rgb}{0.16,0.70,0.27}

\newcommand{\hrefu}[2]{\underline{\href{#1}{#2}}}

\newcommand{\red}[1]{{\color{red} #1}}
\newcommand{\blue}[1]{{\color{blue} #1}}
\newcommand{\darkgreen}[1]{\textcolor{darkgreen}{#1}}
\newcommand{\darkred}[1]{\textcolor{darkred}{#1}}

\newcommand*{\vpointer}{\vcenter{\hbox{\scalebox{1.5}{\large\pointer}}}}

\newcommand{\be}[1]{\begin{equation} \label{#1}}
\newcommand{\ee}{\end{equation}}
\newcommand{\bea}[1]{\begin{eqnarray} \label{#1}}
\newcommand{\eea}{\end{eqnarray}}
\newcommand{\bean}{\begin{eqnarray*}}
\newcommand{\eean}{\end{eqnarray*}}

\newcommand{\non}{\nonumber\\}
\newcommand{\eq}[1]{(\ref{#1})}
\newcommand{\difp}[2]{\frac{\partial #1}{\partial #2}}
\newcommand{\br}{{\bf r}}
\newcommand{\bR}{{\bf R}}
\newcommand{\bA}{{\bf A}}
\newcommand{\bB}{{\bf B}}
\newcommand{\bE}{{\bf E}}
\newcommand{\bm}{{\bf m}}
%\renewcommand{\bm}{{\bf m}}
\newcommand{\bn}{{\bf n}}
\newcommand{\bN}{{\bf N}}
\newcommand{\bp}{{\bf p}}
\newcommand{\bP}{{\bf P}}
\newcommand{\bF}{{\bf F}}
\newcommand{\by}{{\bf y}}
\newcommand{\bz}{{\bf z}}
\newcommand{\bZ}{{\bf Z}}
\newcommand{\bV}{{\bf V}}
\newcommand{\bv}{{\bf v}}
\newcommand{\bu}{{\bf u}}
\newcommand{\bx}{{\bf x}}
\newcommand{\bX}{{\bf X}}
\newcommand{\bW}{{\bf W}}
\newcommand{\bJ}{{\bf J}}
\newcommand{\bj}{{\bf j}}
\newcommand{\bk}{{\bf k}}
\newcommand{\bTheta}{{\bf \Theta}}
\newcommand{\btheta}{{\boldsymbol\theta}}
\newcommand{\bOmega}{{\bf \Omega}}
\newcommand{\bomega}{{\boldsymbol\omega}}
\newcommand{\brho}{{\boldsymbol\rho}}
\newcommand{\rd}{{\rm d}}
\newcommand{\rJ}{{\rm J}}
\newcommand{\ph}{{\varphi}}
\newcommand{\te}{\theta}
\newcommand{\tht}{\vartheta}
\newcommand{\vpar}{v_\parallel}
\newcommand{\vparkb}{v_{\parallel k b}}
\newcommand{\vparkm}{v_{\parallel k m}}
\newcommand{\Jpar}{J_\parallel}
\newcommand{\ppar}{p_\parallel}
\newcommand{\Bpstar}{B_\parallel^*}
\newcommand{\intpi}{\int\limits_{0}^{2\pi}}
\newcommand{\summ}{\sum \limits_{m=-\infty}^\infty}
\newcommand{\tb}{\tau_b(\uv)}
\newcommand{\bh}{{\bf h}}
\newcommand{\cE}{{\cal E}}
\newcommand{\bsigma}{{\boldsymbol\sigma}}
\newcommand{\bS}{{\mathbf S}}
\newcommand{\bI}{{\mathbf I}}
\newcommand{\odtwo}[2]{\frac{\rd #1}{\rd #2}}
\newcommand{\pdone}[1]{\frac{\partial}{\partial #1}}
\newcommand{\pdtwo}[2]{\frac{\partial #1}{\partial #2}}
\newcommand{\ds}{\displaystyle}

%% Titelblatt-Einstellungen
\title[]
{Python 06}
\author[E.~Wachmann]{\scriptsize Elias Wachmann
}
\date{2024} % \today für heutiges Datum verwenden
\institute[Institute of Theoretical and Computational Physics]
{
}
\instituteurl{www.tugraz.at}
% \institutelogo{kurz.pdf}
%~ \additionallogo{merged_logos}
\AtBeginSection[]{
  \begin{frame}
  \vfill
  \centering
  \begin{beamercolorbox}[sep=8pt,center,shadow=true,rounded=true]{title}
    \usebeamerfont{title}\insertsectionhead\par%
  \end{beamercolorbox}
  \vfill
  \end{frame}
}
%%%%%%%%%%%%%%%%%%%%%%%%%%%%%%%%%%%%%%%%%%%%%%%%%%%%%%%%%%%%%%%%%%%%%%%%%%%%
\begin{document}
%%%%%%%%%%%%%%%%%%%%%%%%%%%%%%%%%%%%%%%%%%%%%%%%%%%%%%%%%%%%%%%%%%%%%%%%%%%%
\titleframe

%\begin{frame}
%  \frametitle{Outline}
%  \tableofcontents%[hideallsubsections] 
%  \note{
%  	Meine Präsentation ist wie folgt strukturiert \ldots
%  }
%\end{frame}

\section*{Content}

\begin{frame}
\frametitle{Content}
  \tableofcontents
\end{frame}

%%%%%%%%%%%%%%%%%%%%%%%%%%%%%%%%%%%%%%%%%%%%%%%%%%%%%%%%%%%%%%%%%%%%%%%%%%%%
\section{Indexing revisited}
\begin{frame}
  \frametitle{Logical indexing}
  Using logical indexing, we can select elements of an array that satisfy a certain condition. For example, we can select all elements of an array that are larger than a certain value. The result is a 1D array containing the selected elements. 
  \lstinputlisting[language=python, firstline=3]{examples/logic\_indexing1.py}
\end{frame}

\begin{frame}
  \frametitle{Logical Operators on numpy arrays}
  Numpy arrays can be combined using logical operators: 
  \begin{itemize}
    \item \hrefu{https://numpy.org/doc/stable/reference/routines.logic.html}{\texttt{logical\_and}}
    \item \hrefu{https://numpy.org/doc/stable/reference/routines.logic.html}{\texttt{logical\_or}}
    \item \hrefu{https://numpy.org/doc/stable/reference/routines.logic.html}{\texttt{logical\_not}}
    \item \hrefu{https://numpy.org/doc/stable/reference/routines.logic.html}{\texttt{logical\_xor}}
  \end{itemize}
They evaluate the truth tables (given in Python 05) element-wise.
\end{frame}
\begin{frame}
  \frametitle{Logical Operators on numpy arrays}
  Examples using logical operators:
  \lstinputlisting[language=python, firstline=3,lastline=11]{examples/logic\_indexing2.py}
  \textbf{Note:} B is evaluated on both rows of A.
\end{frame}
\begin{frame}
  \frametitle{Logical Operators on numpy arrays caveats}
 \lstinputlisting[language=python, firstline=5, lastline=5]{examples/logic\_indexing2.py}
  Output:
  \lstinputlisting[language=python, firstline=13, lastline=15]{examples/logic\_indexing2.py}
\end{frame}
\begin{frame}
  \frametitle{np.logical\_and/or/not() vs. \& / \texttt{|} / \texttt{\~{}} }
  While the \hrefu{https://numpy.org/doc/stable/reference/routines.logic.html}{\texttt{logical\_and/or/not}} operators evaluates the truth table element-wise, the binary \hrefu{https://www.w3schools.com/python/gloss_python_bitwise_operators.asp}{\texttt{\&} / \texttt{|} / \texttt{\~{}}} operators evaluates the truth table bitwise.\\
  \textbf{But why should I care?}\\
  Maybe you want to check which elements are not equal to zero between both arrays: 
  \lstinputlisting[language=python, firstline=2, lastline=5]{examples/logic\_indexing3.py}
\end{frame}
\begin{frame}
  \frametitle{np.logical\_and/or/not() vs. \& / \texttt{|} / \texttt{\~{}} }
  \textbf{But why should I care?}
  \lstinputlisting[language=python, firstline=2]{examples/logic\_indexing3.py}
  Remember that the \textbf{bitwise} operators evaluate the truth table \textbf{bitwise}. Therefore, the result is not what we want (or maybe expected).
\end{frame}
\begin{frame}
  \frametitle{np.logical\_and/or/not() vs. \& / \texttt{|} / \texttt{\~{}} }
  \textbf{But why should I care?}
  \lstinputlisting[language=python, firstline=2]{examples/logic\_indexing3.py}
  \vspace{5mm}
  \begin{tabular}{c c c c c c c c c }
    &42: & 0 & 0 & 1 & 0 & 1 & 0 \\ 
    AND&-51:& 1 & 1 & 0 & 0 & 1 & 0 \\\hline
    &8:  & 0 & 0 & 0 & 0 & 1 & 0 \\
  \end{tabular}
\end{frame}

\section{Scope}
\begin{frame}
  \frametitle{Scopes in python}
  Variables in python have a \hrefu{https://www.w3schools.com/python/python_scope.asp}{scope} that defines where they can be accessed.\\
  \lstinputlisting[language=python]{examples/scope1.py}
\end{frame}
\begin{frame}
  \frametitle{Scopes in python}
  A scope is always limited to the current block (/indent).\\
  Variables declared in the outermost scope are called \textbf{global} and can be accessed from anywhere.\\
  \vspace{5mm}
  Variables declared in a function are called \textbf{local} and can only be accessed from within the function.\\
  \vspace{5mm}
  In general, it is good practice to \textbf{avoid} global variables!
\end{frame}
\section{Call by value vs. call by reference}
\begin{frame}
  \frametitle{Call by value vs. call by reference}
  Parameter can be given to functions by value or by reference.\\
  \vspace{5mm}
  Primitive types -- such as \texttt{int}, \texttt{str} and \texttt{float} -- are passed by value (a separate copy is available).\\
  \vspace{5mm}
  Mutable types -- such as \texttt{lists}, \texttt{dictionaries} or \texttt{objects} in general are passed by reference (they refer back to the original). 
\end{frame}
\begin{frame}
  \frametitle{Call by value vs. call by reference}
  \lstinputlisting[language=python]{examples/callrefval.py}
\end{frame}
\section{Copy vs. Deepcopy}
\begin{frame}
  \frametitle{What is a copy?}
  A copy is a new object that is created from an existing object.\\
  \vspace{5mm}
  \textbf{Shallow copy:} A new object is created, but the elements of the new object are references to the elements of the original object.\\
  \vspace{5mm}
  \textbf{Deep copy:} A new object is created, and the elements of the new object are copies of the elements of the original object.
\end{frame}
\begin{frame}
  \frametitle{Create a shallow copy (\hrefu{https://docs.python.org/3/library/copy.html}{copy})}
  Using pythons build-in \hrefu{https://docs.python.org/3/library/copy.html}{copy} module, we can create a shallow copy of an object.\\
  \lstinputlisting[language=python, firstline=1]{examples/copy2.py}
\end{frame}
\begin{frame}
  \frametitle{Create a (deep)copy (\hrefu{https://numpy.org/doc/stable/reference/generated/numpy.copy.html}{np.copy})}
  The whole object can be copied using the \hrefu{https://numpy.org/doc/stable/reference/generated/numpy.copy.html}{copy} function.\\
  \lstinputlisting[language=python, firstline=2]{examples/copy1.py}
  

\end{frame}
\section{Further numpy examples}
\begin{frame}
  \frametitle{Numpy examples -- Reshaping}
  \hrefu{https://numpy.org/doc/stable/reference/generated/numpy.reshape.html}{reshape} can be used to change the shape of an array.
  \lstinputlisting[language=python, firstline=3]{examples/numpy5.py}
\end{frame}
\begin{frame}
  \frametitle{Numpy examples -- Ravel}
  \hrefu{https://numpy.org/doc/stable/reference/generated/numpy.ravel.html}{ravel} can be used to flatten an array.
  \lstinputlisting[language=python, firstline=3]{examples/numpy4.py}
  \hrefu{https://numpy.org/doc/stable/reference/generated/numpy.ndarray.flatten.html}{flatten} is another function that can be used to flatten an array.
\end{frame}


%%%%%%%%%%%%%%%%%%%%%%%%%%%%%%%%%%%%%%%%%%%%%%%%%%%%%%%%%%%%%%%%%%%%%%%%%%%%

\end{document}
%%%%%%%%%%%%%%%%%%%%%%%%%%%%%%%%%%%%%%%%%%%%%%%%%%%%%%%%%%%%%%%%%%%%%%%%%%%%

%% EOF
